\chapter*{Resumen}
\thispagestyle{empty}


La aplicación del aprendizaje reforzado en áreas de control automático representa un paso en la dirección de optimización de tareas que requieren capacidades de autonomía y flexibilidad, rubros que el RL ofrece con métodos como el DQN y el PPO, en su implementación con redes neuronales artificiales. El presente proyecto explora estos métodos en una planta de péndulo amortiguado de motor con hélice, con una estrategia que recompensa el bajo error angular y baja velocidad terminal. Se emplearon entornos virtuales para las pruebas de entrenamiento de control. Se evidencia la necesidad de ajustes en la función de recompensa para dar con el punto de control deseado y acercarse a cumplir con criterios usuales de control.


\bigskip

%% Defina las palabras clave con defKeywords en config.tex:
\textbf{Palabras clave:} \thesisKeywords

\clearpage
\chapter*{Abstract}
\thispagestyle{empty}

The application of reinforced learning in areas of automatic control represents a step in the direction of optimization of tasks that require autonomy and flexibility capabilities, items that RL offers with methods such as DQN and PPO, in its implementation with artificial neural networks. The present project explores these methods in a damped pendulum propeller motor plant, with a strategy that rewards low angular error and low terminal velocity. Virtual environments were used for control training tests. The need for adjustments to the reward function to find the desired control point and come close to meeting the usual control criteria is evident.

\bigskip

\textbf{Keywords:} Reinforcement learning, DQN, PPO, reward function, automatic control

\cleardoublepage

%%% Local Variables: 
%%% mode: latex
%%% TeX-master: "main"
%%% End: 
