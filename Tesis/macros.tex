%%%%%%%%%%%%%%%%%%%%%%%%%%%%%%%%%%%%%%%%%%%%%%%%%%%%%%%%%%%%%%%%%%%%%%%%%%%%%%%
% Author:  Pablo Alvarado
%
% Escuela de Electrónica
% Instituto Tecnológico de Costa Rica
%
% Phone:   +506 2550 9005
% email:   palvarado@itcr.ac.cr
% 
%%%%%%%%%%%%%%%%%%%%%%%%%%%%%%%%%%%%%%%%%%%%%%%%%%%%%%%%%%%%%%%%%%%%%%%%%%%%%%%

\PassOptionsToPackage{usenames,dvipsnames,table}{xcolor}

% -----------------------------------------------------------------------------
%   Define all configuration commands
% -----------------------------------------------------------------------------

\newcommand{\genderLector}[1]{%
  \ifthenelse{\equal{#1}{F}}{%
    Profesora Lectora%
  }{%
    \ifthenelse{\equal{#1}{M}}{%
      Profesor Lector%
    }{%
      Persona profesora lectora%
    }%
  }%
}

% Lector I
\newcommand{\nameLectorI}{$<$\emph{Use defLectorI in config.tex}$>$}
\newcommand{\genderLectorI}{N/A}

\newcommand{\defLectorI}[1][M]{%
  \renewcommand{\genderLectorI}{\genderLector{#1}}%
  \lectorIRelay%
}

\newcommand{\lectorIRelay}[1]{%
  \renewcommand{\nameLectorI}{#1}
}

%% Lector II
\newcommand{\nameLectorII}{$<$\emph{Use setLectorII in main.tex}$>$}
\newcommand{\genderLectorII}{N/A}

\newcommand{\defLectorII}[1][M]{%
  \renewcommand{\genderLectorII}{\genderLector{#1}}%
  \lectorIIRelay%
}

\newcommand{\lectorIIRelay}[1]{%
  \renewcommand{\nameLectorII}{#1}
}

%% Asesor
\newcommand{\nameAsesor}{$<$\emph{Use setAsesor in main.tex}$>$}
\newcommand{\genderAsesor}{$<$\emph{Use setAsesor in main.tex}$>$}

\newcommand{\defAsesor}[1][M]{%
  \ifthenelse{\equal{#1}{F}}{%
    \renewcommand{\genderAsesor}{Profesora Asesora}%
  }{%
    \ifthenelse{\equal{#1}{M}}{%
      \renewcommand{\genderAsesor}{Profesor Asesor}%
    }{%
      % La RAE no ha definido cómo hacer esto...
      %
      % Hay que preguntar a la persona asesora no-binaria directamente
      % cómo gusta ser tratada.
      \renewcommand{\genderAsesor}{Persona profesora asesora}%
    }
  }
  \asesorRelay
}

\newcommand{\asesorRelay}[1]{%
  \renewcommand{\nameAsesor}{#1}
}

% Dates for draft and final
\newcommand{\thesisDraftDate}{\today}
\newcommand{\defDraftDate}[1]{\renewcommand{\thesisDraftDate}{#1}}

\newcommand{\thesisFinalDate}{$<$\emph{Fecha de Defensa en config.tex}$>$}
\newcommand{\defFinalDate}[1]{\renewcommand{\thesisFinalDate}{#1}}

% Author definition and gender related strings

\newcommand{\thesisAuthor}{Error: Undefined}
\newcommand{\thesisAuthorShort}{Error: Undefined}
\newcommand{\thesisAuthorTECID}{Error: Undefined}
\newcommand{\thesisAuthorAddress}{Error: Undefined}
\newcommand{\thesisAuthorDegree}{Error: Undefined}

\newcommand{\defAuthor}[1][M]{%
  \ifthenelse{\equal{#1}{F}}{%
    \renewcommand{\thesisAuthorAddress}{la señora}
    \renewcommand{\thesisAuthorDegree}{Ingeniera}
  }{%
    \ifthenelse{\equal{#1}{f}}{%
      \renewcommand{\thesisAuthorAddress}{la señorita}
      \renewcommand{\thesisAuthorDegree}{Ingeniera}
    }{%
      \ifthenelse{\equal{#1}{M}}{%
        \renewcommand{\thesisAuthorAddress}{el señor}
        \renewcommand{\thesisAuthorDegree}{Ingeniero}
        
      }{%
        \renewcommand{\thesisAuthorAddress}{la persona}
        \renewcommand{\thesisAuthorDegree}{Ingeniere}
      }
    }
  }
  \authorRelay
}

\newcommand{\authorRelay}[1]{%
  \renewcommand{\thesisAuthor}{#1}
}

\newcommand{\defAuthorShort}[1]{%
  \renewcommand{\thesisAuthorShort}{#1}
}

\newcommand{\defAuthorTECID}[1]{%
  \renewcommand{\thesisAuthorTECID}{#1}
}

%% About the institution and department
\newcommand{\thesisDepartment}{Escuela de Ingeniería Electrónica}
\newcommand{\thesisInstitution}{Tecnológico de Costa Rica}

\newcommand{\defDepartment}[1]{%
  \renewcommand{\thesisDepartment}{#1}
}

\newcommand{\defInstitution}[1]{%
  \renewcommand{\thesisInstitution}{#1}
}



%% About the report name, and type
\newcommand{\thesisTitle}{Error: Undefined}
\newcommand{\thesisFlatTitle}{\thesisTitle}
\newcommand{\thesisKeywords}{Error: Undefined}
\newcommand{\thesisType}{Error: Undefined}

%% A tool to remove new lines leaving no spaces
\newcommand\titleFlattener[1]{\def\\{\relax\ifhmode\unskip\fi\space\ignorespaces}#1}

\newcommand{\defTitle}[1]{%
  \renewcommand{\thesisFlatTitle}{\titleFlattener{#1}}
  \renewcommand{\thesisTitle}{#1}
}

\newcommand{\defKeywords}[1]{%
  \renewcommand{\thesisKeywords}{#1}
}

\newcommand{\defTFGType}[1]{%
  \renewcommand{\thesisType}{#1}
}

%% Este archivo contiene toda la configuración básica del documento de
%% tesis, para centralizar alguna información que se requiere en todo
%% el documento.

%% DRAFT MODE

%%   El modo borrador activa las listas de cosas por hacer, con su
%%   índice, y algunas marcas explícitas de "borrador" por todo lado.
%%
%%   Asegúrse de que esta variable sea false en la versión final y de haber
%%   actualizado la fecha de la defensa, un poco más abajo.
\setboolean{draftmode}{true}            % turn draft mode on
%\setboolean{draftmode}{false}          % turn draft mode off

%% Esta es la fecha que se colocará en el modo borrador
\defDraftDate{\today}
%% Esta es la fecha que se usará en la versión final
\defFinalDate{24 de noviembre, 2023}

%% Este es el nombre del estudiante y el pronombre que utiliza, para cambiar
%% las portadas como corresponde

% Con el nombre de autor, se debe especificar el género a utilizar:
%
%   [M]asculino
%   [F]emenino (usando "señora" donde corresponda)
%   [f]emenino (usando "señorita" donde corresponda)
%   persona [N]o binaria
%
%   Debido a la falta de norma en español para las personas no binarias,
%   posíblemente deba ajustarse para el gusto de cada quien.
%
\defAuthor[M]{Oscar Andrés Rojas Fonseca}                % Nombre del estudiante
%\defAuthor[f]{María del Pilar Pérez Prado}    % Nombre de la estudiante

\defAuthorShort{O.~Rojas}                      % Nombre corto
\defAuthorTECID{2018102187}                     % Carné

%% Este es el título completo del informe del trabajo final de graduación.
%% Usted puede agregar \\ para forzar líneas nuevas en la portada y automática-
%% mente el comando se encarga de eliminar eso cuando necesita el título
\defTitle{Método basado en aprendizaje reforzado \\% 
para el control automático de una \\%
planta no lineal}

%% Palabras clave
\defKeywords{palabras, clave, energía, cambio climático, RISC V}

%% Tribunal Evaluador
%%
%% Indique los nombres de los lectores y asesor
%% El parámetro opcional es
%%  [M]asculino,
%%  [F]emenino,
%%  persona [N]o binaria
\defLectorI[F]{Dra.\,María Curie Pérez}
\defLectorII[M]{M.Sc.\,Pedro Pérez Pereira}
\defAsesor[M]{Ing.\,Albert Einstein Sánchez}

%% Tipo de tesis o informe
%%   - Tesis de Licenciatura
%%   - Informe de Proyecto Final 
%%   - Tesis de Maestría
\defTFGType{Tesis de Licenciatura}

%% Nombre del departamento e institución
\defInstitution{Instituto Tecnológico de Costa Rica}
\defDepartment{Escuela de Ingeniería Electrónica}

 % Load the desired configuration

%Para el PDF (cambiar si se desea otras cosas a lo indicado arriba
\newcommand{\pdfAuthor}{\thesisAuthor}
\newcommand{\pdfTitle}{\thesisFlatTitle} 
\newcommand{\pdfKeywords}{\thesisKeywords}
\newcommand{\pdfSubject}{\thesisType}


%% -------------------------------------------------------------------

\usepackage{ifpdf}

% Command to change between draft or release mode:
\newcommand{\ifdraft}[2]{\ifthenelse{\boolean{draftmode}}{#1}{#2}}
% Command to change between draft or release mode:
\newcommand{\ifbook}[2]{\ifthenelse{\boolean{bookmode}}{#1}{#2}}

% include all required packages here
\usepackage{csquotes}                  % recommended for biblatex
\usepackage[spanish,es-tabla]{babel}   % spanish, remove es-tabla for cuadro
%\usepackage[spanish]{babel}   % spanish, remove es-tabla for cuadro

\usepackage{xspace} % Decide if a space is needed at the end of some commands

\makeatletter
% babel uses a hook and therefore the tablename is here not defined yet.
% However, it defines es@tablename with upper/lowercase, and we use it.
\ifthenelse{\equal{\es@tablename}{Ttabla}}{%
  \newcommand{\cuadro}{tabla}
  \newcommand{\cuadros}{tablas}
  \newcommand{\Cuadro}{Tabla}
  \newcommand{\elcuadro}{la tabla}
  \newcommand{\Elcuadro}{La tabla}
  \newcommand{\loscuadros}{las tablas}
  \newcommand{\Loscuadros}{Las tablas}
  
  \newcommand{\tabla}{tabla}
  \newcommand{\tablas}{tablas}
  \newcommand{\Tabla}{Tabla}
  \newcommand{\latabla}{la tabla}
  \newcommand{\Latabla}{La tabla}
  \newcommand{\lastablas}{las tablas}
  \newcommand{\Lastablas}{Las tablas}
  
  \newcommand{\tabref}[1]{\hyperref[#1]{tabla~\ref*{#1}}\xspace}
  \newcommand{\Tabref}[1]{\hyperref[#1]{Tabla~\ref*{#1}}\xspace}

  \newcommand{\latabref}[1]{la \tabref{#1}}
  \newcommand{\Latabref}[1]{La \tabref{#1}}

}{%
  \newcommand{\cuadro}{cuadro}
  \newcommand{\cuadros}{cuadros}
  \newcommand{\Cuadro}{Cuadro}
  \newcommand{\elcuadro}{el cuadro}
  \newcommand{\Elcuadro}{El cuadro}
  \newcommand{\loscuadros}{los cuadros}
  \newcommand{\Loscuadros}{Los cuadros}
  
  \newcommand{\tabla}{cuadro}
  \newcommand{\tablas}{cuadros}
  \newcommand{\Tabla}{Cuadro}
  \newcommand{\latabla}{el cuadro}
  \newcommand{\Latabla}{El cuadro}
  \newcommand{\lastablas}{los cuadros}
  \newcommand{\Lastablas}{Los cuadros}
  
  \newcommand{\tabref}[1]{\hyperref[#1]{cuadro~\ref*{#1}}\xspace}
  \newcommand{\Tabref}[1]{\hyperref[#1]{Cuadro~\ref*{#1}}\xspace}

  \newcommand{\latabref}[1]{el \tabref{#1}}
  \newcommand{\Latabref}[1]{El \tabref{#1}}

}
\makeatother

% References to figures
\newcommand{\figref}[1]{\hyperref[#1]{figura~\ref*{#1}}\xspace}
\newcommand{\Figref}[1]{\hyperref[#1]{Figura~\ref*{#1}}\xspace}
\newcommand{\lafigref}[1]{la \hyperref[#1]{figura~\ref*{#1}}\xspace}
\newcommand{\Lafigref}[1]{La \hyperref[#1]{figura~\ref*{#1}}\xspace}

% References to equations
\newcommand{\equ}[1]{\hyperref[#1]{(\ref*{#1})}\xspace}

% Refrences to chapters and sections
\newcommand{\capref}[1]{\hyperref[#1]{capítulo~\ref{#1}}\xspace}
\newcommand{\secref}[1]{\hyperref[#1]{sección~\ref{#1}}\xspace}

\usepackage{makeidx}                    % to create index file

\usepackage[nottoc]{tocbibind}          % Fix the hyperrefs to TOC,TOF, etc.
                                        % and ensure that they appear all in 
                                        % the Table of Contents
\ifdraft{%
  %\usepackage[refpage]{nomencl}        % Use to easily administrate the list
  \usepackage[intoc,spanish]{nomencl}   % of symbols
}{%
  \usepackage[intoc,spanish]{nomencl}
}

\usepackage{siunitx}                    % Units of the SI
\sisetup{output-decimal-marker = {,}}   % Use decimal , instead of decimal .
\DeclareSIQualifier\peak{p}
\DeclareSIQualifier\ppeak{pp}

\usepackage{amsmath}
\usepackage{amssymb,amstext}            % AMS-math and symbols package
\usepackage{mathrsfs}                   % Calygraphic fonts for transforms
\usepackage{trsym}                      % Para símbolos de transformadas o---o
\usepackage{stmaryrd}                   % For short arrows
\usepackage{nicefrac}
\usepackage{array}                      % extensions to tabular environment
\usepackage{longtable}                  % supports extraordinary long tables
\usepackage{tabularx}                   % supports tables with fixed width

\usepackage[backend=biber,              % Use biber/biblatex
            style=ieee,
            sorting=none,
            citestyle=numeric-comp]{biblatex}

\usepackage{afterpage}                  % put something only after the page
\usepackage{multirow}                   % supports multiple row grouping in 
                                        % tables
\usepackage{multicol}                   % multiple columns environments
\usepackage{enumitem}                   % better enumeration with paralist 
                                        % equivalents as follows:

\newlist{compactitem}{itemize}{3}
\setlist[compactitem]{topsep=0pt,partopsep=0pt,itemsep=0pt,parsep=0pt}
\setlist[compactitem,1]{label=\textbullet}
\setlist[compactitem,2]{label=---}
\setlist[compactitem,3]{label=*}

\newlist{compactdesc}{description}{3}
\setlist[compactdesc]{topsep=0pt,partopsep=0pt,itemsep=0pt,parsep=0pt}

\newlist{compactenum}{enumerate}{3}
\setlist[compactenum]{label*=\arabic*.,topsep=0pt,partopsep=0pt,itemsep=0pt,parsep=0pt}
\newlist{compactenuma}{enumerate}{3}
\setlist[compactenuma]{label*=\alph*.,topsep=0pt,partopsep=0pt,itemsep=0pt,parsep=0pt}

\usepackage{icomma}                     % decimal comma in math mode

\usepackage{bold-extra}

\usepackage[format=hang,%
            font=small,%
            labelfont=bf]{caption}      % nicer figure captions
\usepackage{subcaption}                 % for subfigures
            
\usepackage{booktabs}                   % book type tabulars
\usepackage{pdfpages}                   % used to include the final "acta"

% the own style with options depending on the draft mode
\ifdraft{%
\usepackage[todo]{sty/tecStyle}         % some command definitions
                                        % options [todo] todo-index
}{%
\usepackage{sty/tecStyle}               % some command definitions
                                        % options [todo] todo-index
}

%% fix the title for examples
\renewcommand{\examplelistname}{Índice de ejemplos}
\renewcommand{\examplename}{Ejemplo}

%% define some command to cope with the tribunal names

%% Lector I


\usepackage{url}                        % allows linebreaks at certain
                                        % characters or combinations of 
                                        % characters for URLs

%% Usual tikz libraries and configuration
\usepackage{tikz}
\usepackage{pgfplots}
\pgfplotsset{compat=1.16}
\usepgfplotslibrary{fillbetween}
\usetikzlibrary{patterns,shapes,arrows.meta,positioning,calc,babel}
\usetikzlibrary{fit,shapes.geometric,decorations.markings}

\usepackage{listings}                   % syntax highlighting of code fragments

\lstdefinestyle{verilog-style}
{
  language=Verilog,
  basicstyle=\small\ttfamily,
  keywordstyle=\color{dkblue},
  identifierstyle=\color{black},
  commentstyle=\color{dkgreen},
  numbers=left,
  numberstyle=\tiny\color{black},
  numbersep=10pt,
  tabsize=8,
  moredelim=*[s][\colorIndex]{[}{]},
  literate=*{:}{:}1
}
\lstset{literate=%
  {á}{{\'a}}1
  {é}{{\'e}}1
  {í}{{\'i}}1
  {ó}{{\'o}}1
  {ú}{{\'u}}1
  {ñ}{{\~n}}1
  {Á}{{\'A}}1
  {É}{{\'E}}1
  {Í}{{\'I}}1
  {Ó}{{\'O}}1
  {Ú}{{\'U}}1
  {Ñ}{{\~N}}1
}

\definecolor{vorange}{RGB}{255,143,102}
\definecolor{tableheader}{rgb}{0.2,0.25,0.31}

\makeatletter
\newcommand*\@lbracket{[}
\newcommand*\@rbracket{]}
\newcommand*\@colon{:}
\newcommand*\colorIndex{%
    \edef\@temp{\the\lst@token}%
    \ifx\@temp\@lbracket \color{black}%
    \else\ifx\@temp\@rbracket \color{black}%
    \else\ifx\@temp\@colon \color{black}%
    \else \color{vorange}%
    \fi\fi\fi
}
\makeatother


% For pdflatex
% - The hyperref package should always be loaded last, since it has to
%   overwrite some of the commands.
% - The package subfigure caused that the pagebackrefs and index refs were set
%   incorrectly.

\ifpdf
%
% final / draft document options
%\usepackage{graphicx}                   % for inserting pdf-graphics.
                                        % options final / draft
\ifdraft{%
\usepackage[%pdftex,%
            naturalnames=true,
            linktocpage,
            hyperindex,
            colorlinks,
            urlcolor=dkred,          %\href to external url
            filecolor=dkmagenta,     %\href to local file
            linkcolor=dkred,         %\ref and \pageref
            citecolor=dkgreen,       %\cite
            plainpages=false,
            pdfpagelabels,
            pdfpagemode=UseOutlines, % means use bookmarks (None,UseOutlines)
            % bookmarksopen=false,   % would show the whole hierarchy if true
            bookmarksnumbered=true,
            pdfpagelayout=OneColumn, % SinglePage,OneColumn,TwoColumnLeft,...
            pdfview=FitH, % FitB,FitBH,FitBV,Fit,FitH,FitV
            pdfstartview=FitH, % FitB,FitBH,FitBV,Fit,FitH,FitV
            unicode,
            ]{hyperref}
}{%
% Use biber/biblatex
\usepackage[%pdftex,%
            naturalnames=true,
            linktocpage,hyperindex,
            colorlinks,
            urlcolor=dkred,          %\href to external url
            filecolor=dkmagenta,     %\href to local file
            linkcolor=dkred,         %\ref and \pageref
            citecolor=dkgreen,       %\cite
            plainpages=false,
            pdfpagelabels,
            pdfpagemode=UseOutlines, % means use bookmarks (None,UseOutlines)
            % bookmarksopen=false,   % open the whole hierarchy if true!
            bookmarksnumbered=true,
            pdfpagelayout=OneColumn, % SinglePage,OneColumn,TwoColumnLeft,...
            pdfview=FitH, % FitB,FitBH,FitBV,Fit,FitH,FitV
            pdfstartview=FitH, % FitB,FitBH,FitBV,Fit,FitH,FitV,
            unicode,
            ]{hyperref}
}


%
% Ensure that the links of the images point to the top of the images and not
% to the caption
%
\usepackage[figure]{hypcap}

% %
% % Ensure that pdfLaTeX do the same spacing as LaTeX
% %
\pdfadjustspacing=1 
% %
\else   % i.e. if not pdf

\usepackage[active]{srcltx}             % insert links into the dvi to jump
\usepackage{graphicx}                   % for inserting eps-graphics.
                                        % options final / draft
                                        % into the sources directly.
\ifdraft{%
\usepackage[ps2pdf,%
            % plainpages=false,
            linktocpage,
            hyperindex,
            % pdfpagelabels,
            pdfpagemode=UseOutlines,
            pdfstartview=FitH,
            unicode,]{hyperref}
}{%
\usepackage[ps2pdf,%
            % plainpages=false,
            linktocpage,
            hyperindex,
            % pdfpagelabels,
            pdfpagemode=UseOutlines,
            pdfstartview=FitH,
            unicode,]{hyperref}
}

%\usepackage[ps2pdf]{hyperref}

\fi  % end of if pdf or not

% --------------------------------------------------------------------------

% Allow the use of international characters
\AtBeginDocument{%
  \hypersetup{%
             pdftitle={\pdfTitle},%
             pdfsubject={\pdfSubject},%
             pdfauthor={\pdfAuthor},%
             pdfkeywords={\pdfKeywords}
            }%
}


%\usepackage{algorithmic}            % algorithmic environment


\usepackage{rotating}                % allow block rotation

%% This environment will allow to rotate a page in the PDF, just
%% for visualization purposes.  Since nowadays the documents are
%% almost never printed, it is best if rotated pages are also shown
%% rotated in the PDF viewer, and this is the purpose of this environment.
\newenvironment{rotatepage}%
{\global\pdfpageattr\expandafter{\the\pdfpageattr/Rotate 90}}%
{\clearpage\pagebreak[4]\global\pdfpageattr\expandafter{\the\pdfpageattr/Rotate 0}}

    
%%%%%%%%%%%%%%%%%%%%%%%%%%%%%%%%%%%%%%%%%%%%%%%%%%%%%%%%%%%%%%%%%%%%%%%%%%%%%%%

%\sloppy

%
% Some own font definitions
%
\DeclareMathAlphabet{\mathpzc}{OT1}{pzc}{m}{it}
\DeclareMathAlphabet{\mathpss}{OT1}{cmss}{m}{sl}

%
% page layout
%

\usepackage{vmargin}
\setpapersize{USletter}

% For letter-paper printing
\setmarginsrb{33mm}{8mm}{23mm}{7mm}{15pt}{15pt}{7mm}{12mm}
%\setlength{\headheight}{15pt}         % fancy headers wanted this

%
% Fraction of Float Object / Text
%

\renewcommand{\topfraction}{0.95}       % how much of top of page should be 
                                        % allowed to be float object?
\renewcommand{\bottomfraction}{0.95}    % how much of bottom of page should be
                                        % allowed to be float object?
\renewcommand{\textfraction}{0.05}      % how much of page must be text?

\usepackage{fancyhdr}                   % fancy page headers

\usepackage{lastpage}

%
% header and footer layout (needs package fancyhdr)
%
\newcommand{\copyrightfooter}{\tiny{\copyright \the\year\ --- \thesisAuthorShort %
    \qquad Uso exclusivo ITCR}}
%
\newcommand{\draftfoot}%
  {\ifdraft{\textcolor{dkblue}{\tiny\textsl{Borrador: \today}}{}}
           {}
}

\pagestyle{fancy}
\renewcommand{\chaptermark}[1]{\markboth{{\small
    \thechapter\hspace*{1mm}#1}}{}}
\renewcommand{\sectionmark}[1]{\markright{{\small
    \thesection\hspace*{1mm}#1}}{}}
%\lhead[{\small\textsc\Roman{\thepage}}]{\fancyplain{}%
\lhead[{\small\thepage}]{\fancyplain{}%
        {{\slshape \small\nouppercase{\leftmark}}}}
\chead[]{}
\rhead[\fancyplain{}%
%        {{\slshape \small\nouppercase{\rightmark}}}]{{\small\textsc\Roman{\thepage}}}
        {{\slshape \small\nouppercase{\rightmark}}}]{{\small\thepage}}
\lfoot[]{\draftfoot}
\ifbook{%
  \cfoot[]{}
}{
  \cfoot[\copyrightfooter]{\copyrightfooter}
}
\rfoot[\draftfoot]{}
\renewcommand{\headrulewidth}{0.5pt}
\renewcommand{\footrulewidth}{0pt}

%
% Caption style for tables
% For caption v3:
\captionsetup[table]{position=top,
  format=hang,
  textfont={normalsize},
  labelfont={normalsize,bf}}

\newcommand{\tablecaption}[2][foo]{%
  \ifthenelse{\equal{#1}{foo}}{%
    \caption{#2}%
  }{%
    \caption[#1]{#2}%
  }
}

%% En español hay diferencia entre Tabla y Cuadro.
%%
%% Tabla: es la tabla periódica, o una tabla de logaritmos o de
%%        probabilidades o de integrales. Usualmente es algo que no
%%        requiere referencias o explicaciones adicionales en el texto porque
%%        todas sus entradas son sucesiones de alguna cosa que se busca allí
%%        mismo
%% Cuadro: es lo que usualmente se usa en inglés como "table", y resume
%%        información que requiere al menos parcialmente explicaciones en el
%%        el texto.

%% \addto\extrasspanish{\renewcommand{\tablename}{Tabla}}
%% \addto\extrasspanish{\renewcommand{\listtablename}{\'Indice de tablas}}

%
% paragraph layout
%
\renewcommand{\baselinestretch}{1.1}    % line spacing
\parindent0em                           % indentation width of first line
\parskip1.3ex                           % space between paragraphs

%
% document consists of
% chapter - section - subsection - subsubsection - paragraph - subparagraph
%
\setcounter{secnumdepth}{2}             % depth of section numbering
\setcounter{tocdepth}{2}                % depth of table of contents

% For biblatex
\addbibresource{literatura.bib}

%
% prepares index from entries like \index{word} or \index{group!word}.
% don't forget to call "makeindex filename" for final index generation.
%
\makeindex                            %% for package makeidx.sty
%\newindex{default}{idx}{ind}{Index}  %% for package index.sty

\newcommand{\octave}{GNU/Octave}
\newcommand{\linux}{GNU/Linux}


%
% prepares notation or nomenclature 
%
\makenomenclature

%%% Local Variables: 
%%% mode: latex
%%% TeX-master: "main"
%%% End: 

%**********************************************************
%**********************************************************

\usepackage{comment}

\usepackage{algorithm}
\floatname{algorithm}{Algoritmo}
\renewcommand{\listalgorithmname}{Índice de Algoritmos}
\usepackage{algpseudocode}

