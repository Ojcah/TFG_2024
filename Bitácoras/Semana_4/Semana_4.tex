\documentclass[12pt]{article}
\usepackage{fullpage,amsmath,amssymb,graphicx}

\usepackage{setspace}
\spacing{1}

\usepackage{textpos}
\usepackage{tikz}
\usepackage{pgf}
\usepackage{amssymb}
\usepackage{enumerate}
\usepackage{xcolor}
\usepackage{graphicx}
\usepackage{subcaption}
\usepackage{tabularx}
\usepackage{colortbl}
\usepackage{multicol}
\usepackage{longtable}
\usepackage{hyperref}


\definecolor{encabezado}{rgb}{0.74, 0.83, 0.9}

\begin{document}

\hfill\\
\rule{\textwidth}{1.5pt}

\begin{minipage}[t]{85mm}
  \begin{tabular}{l}
    \textbf{\large Instituto Tecnológico de Costa Rica} \\  
    \textbf{Escuela de Ingeniería Electrónica} \\
    \textbf{Trabajo Final de Graduación} \\
    \textbf{Proyecto:} Método basado en aprendizaje reforzado \\para el control automático de una planta no lineal. \\
    \textbf{Estudiante:} Oscar Andrés Rojas Fonseca \hspace{3cm}\rule{4.5cm}{1.5pt}\\
    I Semestre 2024 \hspace{8.5cm}\textbf{Firma del asesor}
  \end{tabular}
\end{minipage}
\hfill\\
\rule{\textwidth}{1.5pt}


\section*{Bitácora de trabajo}

%\begin{table}[h]
\begin{minipage}[h]{\textwidth}
	\centering
	\begin{tabularx}{\textwidth}{|p{2cm}|X|X|p{2cm}|} 
		\hline
		\rowcolor{encabezado}
		\textbf{Fecha} & 
		\textbf{Actividad} & 
		\textbf{Anotaciones} & 
		\textbf{Horas dedicadas} \\ \hline
		% ***************************************************************
		26/02/2024 & 
		$\mathbf{1}.$ Pruebas de entrenamiento del modelo imitador en Nvidia K80. & 
		$a)$ AAAAAAAAAAAAAA \newline  & 
		4 horas \\
	 	% ***************************************************************
	 	27/02/2024 & 
	 	$\mathbf{2}.$ ADADADADAD. &
	 	$a)$ AAAAAAAAAAAAAA \newline  & 
	 	5 horas \\
	 	% ***************************************************************
	 	28/02/2024 & 
	 	$\mathbf{3}.$ Estudio de la comunicación entre el sistema (planta) y el módulo de control al sistema (Red neuronal). & 
	 	$a)$ Revisión de la teoría correspondiente en \cite{DataScience}. \newline 
	 	$b)$ ADADADADAd \newline & 
	 	3 horas \\
	 	
	 	\hline
	\end{tabularx}
\end{minipage}	 	
	 	
	 	% ***************************************************************
\hfill\\
\begin{minipage}[h]{\textwidth}
	\centering
	\begin{tabularx}{\textwidth}{|p{2cm}|X|X|p{2cm}|} 
		\hline		
		
	 	29/02/2024 & 
	 	$\mathbf{4}.$ ADADADADAD. & 
	 	$a)$ AAAAAAAAAAAAAA \newline  & 
	 	6 horas \\
	 	% ***************************************************************
	 	01/03/2024 & 
	 	$\mathbf{5}.$ Prueba de entrenamiento con datos reales del PAMH. & 
	 	$a)$ Se ejecutó el script $RNAM\_Real.py$ con una primera versión de los datos recolectados. Sin exito por tiempo de ejecución muy largo. \newline 
	 	$b)$ ADADADA. \newline & 
	 	3 horas \\
	 	% ***************************************************************
	 	\hline
		\multicolumn{3}{|r|}{Total de horas de trabajo:} & 21 horas \\ 
	 	\hline                 
	\end{tabularx}
\end{minipage}
%\end{table}


\begin{enumerate}
	\item Estudio de la comunicación entre el sistema (planta) y el módulo de control al sistema (Red neuronal).
	\begin{enumerate}
		\item Model Predictive Control (MPC) \cite{DataScience} Figura 10.2.
		\item a
	\end{enumerate}
	\item a
	\item a
	\item a
\end{enumerate}


% *****************************************************************************
% *****************************************************************************
% *****************************************************************************
\newpage

\section*{Contenidos de actividades}

\subsection*{Resumen de repositorios encontrados}

ADADADADADAD \cite{DataScience}.






\newpage

\section*{Referencias}
\renewcommand\refname{}
\bibliographystyle{IEEEtran}
\bibliography{references}





\end{document}