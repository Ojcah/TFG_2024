\documentclass[12pt]{article}
\usepackage{fullpage,amsmath,amssymb,graphicx}

\usepackage{setspace}
\spacing{1}

\usepackage{amssymb, amsmath}
\usepackage{textpos}
\usepackage{tikz}
\usepackage{pgf}
\usepackage{amssymb}
\usepackage{enumerate}
\usepackage{xcolor}
\usepackage{graphicx}
\usepackage{subcaption}
\usepackage{tabularx}
\usepackage{colortbl}
\usepackage{multicol}
\usepackage{longtable}


\definecolor{encabezado}{rgb}{0.74, 0.83, 0.9}

\begin{document}

\hfill\\
\rule{\textwidth}{1.5pt}

\begin{minipage}[t]{85mm}
  \begin{tabular}{l}
    \textbf{\large Instituto Tecnológico de Costa Rica} \\  
    \textbf{Escuela de Ingeniería Electrónica} \\
    \textbf{Trabajo Final de Graduación} \\
    \textbf{Proyecto:} Método basado en aprendizaje reforzado \\para el control automático de una planta no lineal. \\
    \textbf{Estudiante:} Oscar Andrés Rojas Fonseca \hspace{3cm}\rule{4.5cm}{1.5pt}\\
    I Semestre 2024 \hspace{8.5cm}\textbf{Firma del asesor}
  \end{tabular}
\end{minipage}
\hfill\\
\rule{\textwidth}{1.5pt}


\section*{Bitácora de trabajo}

%\begin{table}[h]
\begin{minipage}[h]{\textwidth}
	\centering
	\begin{tabularx}{\textwidth}{|p{2cm}|X|X|p{2cm}|} 
		\hline
		\rowcolor{encabezado}
		\textbf{Fecha} & 
		\textbf{Actividad} & 
		\textbf{Anotaciones} & 
		\textbf{Horas dedicadas} \\ \hline
		% ***************************************************************
		05/02/2024 & 
		$\mathbf{1}.$ Estudio de a teoría de control para el péndulo amortiguado a hélice (PAMH). & 
		$a)$ Consulta a bibliografía de control automático: Nise (2020) y Ogata (2003). \newline $b)$ Revisión de material multimedia de Anibal Ruiz-Barquero referente al PAMH vía Youtube. \newline & 
		3 horas \\
	 	% ***************************************************************
	 	06/02/2024 & 
	 	$\mathbf{2}.$ Estudio de la teoría de aprendizaje reforzado (RL). &
	 	$a)$ Consulta a libros de texto como \textit{Data-Driven Science and Engineering} (Brunton y Kutz, 2021). \newline $b)$ Revisión de material multimedia de Steven Brunton vía Youtube. \newline & 
	 	3 horas \\
	 	% ***************************************************************
	 	06/02/2024 & 
	 	$\mathbf{3}.$ Revisión bibliográfica de algoritmos de aplicación de aprendizaje automático. & 
	 	$a)$ Consulta al libro \textit{Reinforcement Learning: An introduction} (Sutton y Barto, 2020) para mayor detalle. \newline $b)$ Revisión de otros métodos de aprendizaje automático. \newline $c)$ Ejemplos de implementación de las redes neuronales recurrentes (RNN) por Patrick Loeber vía Youtube y la tésis de graduación de Jorge Brenes. & 
	 	3 horas \\
	 	
	 	\hline
	\end{tabularx}
\end{minipage}	 	
	 	
	 	% ***************************************************************

\begin{minipage}[h]{\textwidth}
	\centering
	\begin{tabularx}{\textwidth}{|p{2cm}|X|X|p{2cm}|} 
		\hline		
		
	 	07/02/2024 & 
	 	$\mathbf{4}.$ Revisión de repositorios en línea de métodos de aplicación de aprendizaje automático. & 
	 	$a)$ Búsqueda preliminar de repositorios generales de RL mediante Github. \newline $b)$ Selección de códigos con enfoques similares al control del PAMH. \newline & 
	 	5 horas \\
	 	% ***************************************************************
	 	09/02/2024 & 
	 	$\mathbf{5}.$ Creación del ambiente de trabajo anaconda para montaje de la red neuronal mimetizadora (RNAM). & 
	 	$a)$ Revisión de bibliotecas utilizadas por el código base de la RNAM.\newline $b)$ Instalación/revisión de versiones adecuadas de \textit{Python}, \textit{ArgumentParser}, \textit{Numpy}, \textit{Matplotlib}, \textit{TensorFlow} y \textit{Weights\&Biasis}.\newline & 
	 	2 horas \\
	 	% ***************************************************************
	 	09/02/2024 & 
	 	$\mathbf{6}.$ Pruebas de funcionamiento de la red neuronal mimetizadora (Synthetic-PAHM.py). & 
	 	$a)$ Estudio del código de la RNAM. \newline $b)$ Error en el proceso por falta de cuenta y permisos del autor en W\&B. \newline $c)$ Creación de cuenta y proyecto en W\&B. \newline & 
	 	3 horas \\
	 	% ***************************************************************
	 	09/02/2024 & 
	 	$\mathbf{7}.$ Estudio del funcionamiento de la herramienta Weights \& Biases (W\&B). & 
	 	$a)$ Revisión de material en línea sobre el uso de W\&B. \newline $b)$ Ejemplos de implementación de W\&B. & 
	 	2 horas \\
	 	% ***************************************************************
	 	\hline
		\multicolumn{3}{|r|}{Total de horas de trabajo:} & 21 horas \\ 
	 	\hline                 
	\end{tabularx}
\end{minipage}
%\end{table}

\section*{Contenidos de actividades}

\subsection*{Resumen de teoría PAMH}

AAAA

\subsection*{Resumen de teoría RL}

AAAA

\subsection*{Ambiente de trabajo anaconda para la RNAM}

AAAA




\end{document}