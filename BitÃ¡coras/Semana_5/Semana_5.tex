
\documentclass[12pt]{article}
\usepackage{fullpage,amsmath,amssymb,graphicx}

\usepackage{setspace}
\spacing{1}

\usepackage{textpos}
\usepackage{tikz}
\usepackage{pgf}
\usepackage{amssymb}
\usepackage{enumerate}
\usepackage{xcolor}
\usepackage{graphicx}
\usepackage{subcaption}
\usepackage{tabularx}
\usepackage{colortbl}
\usepackage{multicol}
\usepackage{longtable}
\usepackage{hyperref}


\definecolor{encabezado}{rgb}{0.74, 0.83, 0.9}

\begin{document}

\hfill\\
\rule{\textwidth}{1.5pt}

\begin{minipage}[t]{85mm}
  \begin{tabular}{l}
    \textbf{\large Instituto Tecnológico de Costa Rica} \\  
    \textbf{Escuela de Ingeniería Electrónica} \\
    \textbf{Trabajo Final de Graduación} \\
    \textbf{Proyecto:} Método basado en aprendizaje reforzado \\para el control automático de una planta no lineal. \\
    \textbf{Estudiante:} Oscar Andrés Rojas Fonseca \hspace{3cm}\rule{4.5cm}{1.5pt}\\
    I Semestre 2024 \hspace{8.5cm}\textbf{Firma del asesor}
  \end{tabular}
\end{minipage}
\hfill\\
\rule{\textwidth}{1.5pt}


\section*{Bitácora de trabajo}

%\begin{table}[h]
\begin{minipage}[h]{\textwidth}
	\centering
	\begin{tabularx}{\textwidth}{|p{2cm}|X|X|p{2cm}|} 
		\hline
		\rowcolor{encabezado}
		\textbf{Fecha} & 
		\textbf{Actividad} & 
		\textbf{Anotaciones} & 
		\textbf{Horas dedicadas} \\ \hline
		% ***************************************************************
		05/03/2024 & 
		$\mathbf{1}.$ Continuación de pruebas para solucionar el error de OpenGL. & 
		$a)$ Luego de varias actualizaciones de drivers e instalación de los drivers de Nvidia necesarios el error de OpenGL persistió. \newline 
		$b)$ Se optó por utilizar la herramienta \textit{Jupyter Notebook} que funcionó sin errores. \newline & 
		5 horas \\
	 	% ***************************************************************
	 	06/03/2024 & 
	 	$\mathbf{2}.$ Estudio del código de ejemplo \textit{dpg.py} y \textit{q-learning.py} del repositorio \href{https://github.com/FilippoAiraldi/mpc-reinforcement-learning/tree/main}{MPCRL} \cite{Airdaldi2023}. & 
	 	$a)$ Se identificaron las 2 clases principales del código ejemplo, correspondientes a un entorno definido en el momento $LtiSystem$ y una clase de administración de la anterior mediante MPCRL llamada $LinearMpc$. \newline
	 	$b)$ La clase $LinearMpc$ se encarga de hacer la distinción, dentro de sus funciones, entren el método $DPG$ y $Q-Learning$. \newline & 
	 	6 horas \\
	 	
	 	\hline
	\end{tabularx}
\end{minipage}	 	
	 	
	 	% ***************************************************************
\hfill\\
\begin{minipage}[h]{\textwidth}
	\centering
	\begin{tabularx}{\textwidth}{|p{2cm}|X|X|p{2cm}|} 
		\hline		
		
	 	07/03/2024 & 
	 	$\mathbf{3}.$ Pruebas de definición de clases adaptadas en ejemplos de MPCRL a entornos Gymnasium \textit{CartPole} y \textit{Pendulum} \cite{gym}. & 
	 	$a)$ La definición del entorno propio del ejemplo es un poco diferente comparado con la respectiva al $CartPole$ de Gymnasium \cite{gym}.  \newline & 
	 	5 horas \\
	 	% ***************************************************************
	 	07/03/2024 & 
	 	$\mathbf{4}.$ Ordenamiento del material agregado al documento final del proyecto (tesis). &
	 	$a)$ Se copió información anteriormente adquirida en marco teórico. \newline
	 	$b)$ Supresión de líneas previas de la plantilla. \newline & 
	 	3 horas \\
	 	% ***************************************************************
	 	08/03/2024 & 
	 	$\mathbf{5}.$ Estudio de las clases adaptadas $CartPoleSystem$ y $CartPoleMpc$. &
	 	$a)$ Al analizar las estructuras generadas se obtienen errores y parámetros que no corresponde. Es necesario replantear las clases. \newline & 
	 	4 horas \\
	 	% ***************************************************************
	 	
	 	\hline
		\multicolumn{3}{|r|}{Total de horas de trabajo:} & 23 horas \\ 
	 	\hline                 
	\end{tabularx}
\end{minipage}
%\end{table}



% *****************************************************************************
% *****************************************************************************
% *****************************************************************************

\section*{Contenidos de actividades}

La estructura de clases utilizada en los ejemplos responde en primera instancia a una clase llamada en $LtiSystem$, encargada de definir las características del entorno basado en Gymnasium ($step$, $action$ y $get_stage_cost$) que se utiliza para los ejemplos, esto va desde las dimensiones del sistema de espacio de estados hasta los contenidos de sus matrices $A$ y $B$ \cite{Airdaldi2023}.

Seguidamente se cuenta con la clase $LinearMpc$, la cual cuenta con los ajustes referentes a los parámetros del entrenamiento del modelo propio del ejemplo como $V0$, sin embargo, la definición de dichos parámetros para el nuevo entorno $CartPole$ puede recibir ciertas variaciones. Ademaś, esta clase se encarga de la dinámica deseada del entorno y su comunicación para el entrenamiento siguiente \cite{Airdaldi2023}.

Estas características son las que se deben adaptar en la nueva versión del código, primero con los entornos virtuales de Gymnasium y seguidamente, en el control del modelo imitador del PAMH objetivo.





\newpage

\section*{Referencias}
\renewcommand\refname{}
\bibliographystyle{IEEEtran}
\bibliography{references}





\end{document}
