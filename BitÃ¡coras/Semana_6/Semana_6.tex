
\documentclass[12pt]{article}
\usepackage{fullpage,amsmath,amssymb,graphicx}

\usepackage{setspace}
\spacing{1}

\usepackage{textpos}
\usepackage{tikz}
\usepackage{pgf}
\usepackage{amssymb}
\usepackage{enumerate}
\usepackage{xcolor}
\usepackage{graphicx}
\usepackage{subcaption}
\usepackage{tabularx}
\usepackage{colortbl}
\usepackage{multicol}
\usepackage{longtable}
\usepackage{hyperref}
\usepackage{comment}
\usepackage{listings}



\definecolor{encabezado}{rgb}{0.74, 0.83, 0.9}

\begin{document}

\hfill\\
\rule{\textwidth}{1.5pt}

\begin{minipage}[t]{85mm}
  \begin{tabular}{l}
    \textbf{\large Instituto Tecnológico de Costa Rica} \\  
    \textbf{Escuela de Ingeniería Electrónica} \\
    \textbf{Trabajo Final de Graduación} \\
    \textbf{Proyecto:} Método basado en aprendizaje reforzado \\para el control automático de una planta no lineal. \\
    \textbf{Estudiante:} Oscar Andrés Rojas Fonseca \hspace{3cm}\rule{4.5cm}{1.5pt}\\
    I Semestre 2024 \hspace{8.5cm}\textbf{Firma del asesor}
  \end{tabular}
\end{minipage}
\hfill\\
\rule{\textwidth}{1.5pt}


\section*{Bitácora de trabajo}

%\begin{table}[h]
\begin{minipage}[h]{\textwidth}
	\centering
	\begin{tabularx}{\textwidth}{|p{2cm}|X|X|p{2cm}|} 
		\hline
		\rowcolor{encabezado}
		\textbf{Fecha} & 
		\textbf{Actividad} & 
		\textbf{Anotaciones} & 
		\textbf{Horas dedicadas} \\ \hline
		% ***************************************************************
	 	11/03/2024 & 
	 	$\mathbf{1}.$ Pruebas con las versiones de las clases $CartPoleSystem$ y $CartPoleMpc$. & 
	 	$a)$ Luego del nuevo estudio del código original de ejemplo \cite{Airdaldi2023}, se evidencia que las condiciones o parámetros definicos no son adecuados. \newline & 
	 	3 horas \\
		% ***************************************************************
	 	12/03/2024 & 
	 	$\mathbf{2}.$ Continuación de pruebas para la definición de las clases referentes a $CartPole$. & 
	 	$a)$ Se definió una nueva versión de la clase $CartPoleSystem$ utilizando las funciones $\_init\_()$, $reset()$, $step()$, $render()$, $close()$ y $get\_stage\_cost()$. \newline
	 	$b)$ Revisión de parámetros funcionales de los ejemplos con la adaptación a $CartPole$. \newline & 
	 	6 horas \\
		% ***************************************************************
	 	13/03/2024 & 
	 	$\mathbf{3}.$ Reunión de seguimiento con el asesor del proyecto. & 
	 	$a)$ Revisión de avance en el código y resaltado de errores de forma.  \newline
	 	$b)$ Se indicó los puntos importantes por replantear.  \newline & 
	 	2 horas \\
	 	% ***************************************************************
	 	13/03/2024 & 
	 	$\mathbf{4}.$ Continuación de pruebas para la definición de las clases referentes a CartPole. & 
	 	$a)$ Prueba con resultado incorrecto de las matrices que definen la dinámica del sistema. \newline & 
	 	4 horas \\
	 	
	 	\hline
	\end{tabularx}
\end{minipage}	 	
	 	
	 	% ***************************************************************
\hfill\\
\begin{minipage}[h]{\textwidth}
	\centering
	\begin{tabularx}{\textwidth}{|p{2cm}|X|X|p{2cm}|} 
		\hline		
		
		% ***************************************************************
	 	14/03/2024 & 
	 	$\mathbf{5}.$ Reunión de seguimiento con coordinador del TFG a las $15:00$. &
	 	$a)$ No asistí a la reunión por motivos personales. \newline 
	 	$b)$ Se contactó al coordinador mediante correo electrónico para solicitar información. \newline & 
	 	1 horas \\
		% ***************************************************************
		15/03/2024 & 
	 	$\mathbf{6}.$ Hora consulta con coordinador del TFG $08:00$. &
	 	$a)$ Se consultó mediante llamada telefónica para solicitud de información respecto a la reunión. \newline
	 	$b)$ Redacción de correo electrónico con datos solicitados por el coordinador. \newline & 
	 	1 horas \\
	 	% ***************************************************************
	 	15/03/2024 & 
	 	$\mathbf{7}.$ Continuación de pruebas para la definición de las clases referentes a CartPole. &
	 	$a)$ Los parámetros del ejemplo $dpg.py$ son diferentes al caso del $CartPole$ por el objetivo de aprendizaje del modelo. \newline
	 	$b)$ Se redefinieron las funciones $\_init\_()$, $dynamics()$, $plan()$ y $execute()$ de la clase $CartPoleMpc$. \newline
	 	$c)$ Nuevamente se tienen errores de definición, se debe replantear la clase $CartPoleMpc$. \newline & 
	 	6 horas \\
	 	% ***************************************************************
	 	
	 	\hline
		\multicolumn{3}{|r|}{Total de horas de trabajo:} & 23 horas \\ 
	 	\hline                 
	\end{tabularx}
\end{minipage}
%\end{table}



% *****************************************************************************
% *****************************************************************************
% *****************************************************************************

\section*{Contenidos de actividades}

Al observar los ejemplos del repositorio \href{https://github.com/FilippoAiraldi/mpc-reinforcement-learning/tree/main}{$MPC-RL$} \cite{Airdaldi2023} y al replantear las clases $CartPoleSystem$ y $CartPoleMpc$, se definió nuevamente la primera con la estructura siguiente:\\

\textbf{CartPoleSystem(gym.Env):}
\begin{itemize}
	\item $\_init\_(self)$: Define el tipo de \textit{environment} virtual de $Gymnasium$ y guarda las obseraciones del espacio de estados y las acciones.
	\item $reset(self)$: Reinicia el proceso del ambiente virtual.
	\item $step(self, action)$: Recibe la acción, la aplica al ambiente y da un paso en el proceso.
	\item $render(self, mode)$: Plantea el tipo de proceso a observar entre gráfico ($human$) y numérico ($rgb\_array$).
	\item $close(self)$: Termina el proceso del ambiente virtual.
	\item $get\_stage\_cost(self, state)$: Define un primer comportamiento del ambiente virtual para pantearlo a la recompensa.
\end{itemize}

Por otro lado, la clase $CartPoleMpc$ presenta muchos errores, por lo que se debe analizar cuidadosamente y replantear por completo.

\newpage

\section*{Referencias}
\renewcommand\refname{}
\bibliographystyle{IEEEtran}
\bibliography{references}





\end{document}
